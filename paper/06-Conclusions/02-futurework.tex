This framework for LFA of arbitrary order finite element discretizations has several opportunities for continued research.

The work in this dissertation focused on LFA of nodal finite element discretizations where all fields used the same discretization.
One natural extension of this work is the LFA of PDE discretized with mixed finite element methods, such as those seen in solid mechanics applications with a continuous displacement space and a discontinuous pressure space.
Another natural extension is the LFA of PDE with modal or hierarchical bases, where the basis is not associated with nodal locations.

We described the LFA of $p$-multigrid and $h$-multigrid methods.
This framework can also analyze $hp$-multigrid methods, where the prolongation operator is again defined by the evaluation of the coarse grid basis on the fine grid macro-element space.

Currently, the LFA of BDDC is restricted to primal spaces only consisting of subdomain vertices, but it is common to augment these points with subdomain edge and face averages.
Extending the LFA to use richer primal spaces would allow for more general and practical analysis.

Another popular family of domain decomposition method is Schwartz methods, such as additive Schwartz.
Extending this LFA framework to analyze Schwartz methods, or other overlapping domain decomposition techniques, would help quantify convergence differences between overlapping methods and BDDC and facilitate a broader comparison between these two types of methods in terms of both convergence rates and global communication required.
