For high-order matrix-free BDDC, we decompose the domain $\Omega$ into a series of subdomains $\Omega^e$ given by the individual elements and we utilize four non-exclusive spaces on each subdomain.
The interface between subdomains is given by $\Gamma = \bigcup \partial \Omega^e \backslash \partial \Omega$ and the interface of subdomain $\Omega^e$ is given by $\Gamma^e = \delta \Omega^e \bigcap \Gamma$.
The interior of each subdomain, $\text{I}$, is given by the remaining degrees of freedom on the element.

Each subdomain problem could therefore be written as
\begin{equation}
{\color{burgundy}\mathbf{A}}^e =
\left[ \begin{array}{c c}
{\color{burgundy}\mathbf{A}}_{\text{I}, \text{I}}^e  &  {\color{burgundy}\mathbf{A}}_{\Gamma, \text{I}}^{e, T}  \\
{\color{burgundy}\mathbf{A}}_{\Gamma, \text{I}}^e    &  {\color{burgundy}\mathbf{A}}_{\Gamma, \Gamma}^e         \\
\end{array} \right]
\end{equation}
where ${\color{burgundy}\mathbf{A}}^e = {\color{blue(ncs)}\mathbf{B}}^T {\color{applegreen}\mathbf{D}} {\color{blue(ncs)}\mathbf{B}}$, as shown in Equation \ref{eq:localoperator}.

This subdomain problem is assembled into the global problem in the typical finite element approach given by Equation \ref{eq:libceed_representation}, but in BDDC we create a subassembled problem that is easier to invert than the global problem by duplicating broken degrees of freedom along the subdomain boundaries.
Only the corner, or vertex, degrees of freedom from each element are assembled into a global coarse problem while the remaining degreees of freedom on the subdomain interfaces are duplicated in each subdomain.
The solutions on the global coarse problem and broken subdomain problems are used to construct an approximate solution to the true assembled global problem.

The coarse grid, or primal, nodes $\Pi^e$ are given by the corners of the subdomain, so we have $4$ primal nodes for each element in two dimensions and $8$ primal nodes in three dimensions.
The remainder of the subdomain we denote with an $\text{r}$.
The subassembled problem is therefore given by
\begin{equation}
\hat{\color{burgundy}\mathbf{A}} = \sum_{e = 1}^N \mathbf{R}^{e, T} \hat{\color{burgundy}\mathbf{A}}^e \mathbf{R}^e
\label{eq:subassembled}
\end{equation}
where
\begin{equation}
\hat{\color{burgundy}\mathbf{A}}^e =
\left[ \begin{array}{c c}
{\color{burgundy}\mathbf{A}}_{\text{r}, \text{r}}^e  &  \hat{\color{burgundy}\mathbf{A}}_{\Pi, \text{r}}^{e, T}  \\
\hat{\color{burgundy}\mathbf{A}}_{\Pi, \text{r}}^e   &  \hat{\color{burgundy}\mathbf{A}}_{\Pi, \Pi}^e            \\
\end{array} \right]
\end{equation}
and $\mathbf{R}^e$ is the subdomain injection operator. 
In this formulation, the broken degrees of freedom found on the portion of subdomain interface given by $\Gamma^e - \Pi^e$ are duplicated for each subdomain.
This duplication reduces the amount of global communication required to solve the subassemebled problem, which makes BDDC attractive as a preconditioner for high-order finite element methods.

\begin{definition}
The action of the BDDC preconditioner is given by
\begin{equation}
\mathbf{M}^{-1} = \mathbf{R}^T \hat{\color{burgundy}\mathbf{A}}^{-1} \mathbf{R}
\end{equation}
where $\mathbf{R}$ is the subassembly injection operator.
The inverse of the subassembled problem is computed by Schur complement
\begin{equation}
\hat{\color{burgundy}\mathbf{A}}^{-1} =
\left[ \begin{array}{c c}
\mathbf{I}  &  -{\color{burgundy}\mathbf{A}}_{\text{r}, \text{r}}^{-1} \hat{\color{burgundy}\mathbf{A}}_{\Pi, \text{r}}^T  \\
\mathbf{0}  &  \mathbf{I}                                                                                                  \\
\end{array} \right]
\left[ \begin{array}{c c}
{\color{burgundy}\mathbf{A}}_{\text{r}, \text{r}}^{-1}  &  \mathbf{0}                   \\
\mathbf{0}                                              &  \hat{\mathbf{S}}_{\Pi}^{-1}  \\
\end{array} \right]
\left[ \begin{array}{c c}
\mathbf{I}                                                                                                &  \mathbf{0}  \\
-\hat{\color{burgundy}\mathbf{A}}_{\Pi, \text{r}} {\color{burgundy}\mathbf{A}}_{\text{r}, \text{r}}^{-1}  &  \mathbf{I}  \\
\end{array} \right]
\end{equation}
where $\hat{\mathbf{S}}_{\Pi} = \hat{\color{burgundy}\mathbf{A}}_{\Pi, \Pi} - \hat{\color{burgundy}\mathbf{A}}_{\Pi, \text{r}} {\color{burgundy}\mathbf{A}}_{\text{r}, \text{r}}^{-1} \hat{\color{burgundy}\mathbf{A}}_{\Pi, \text{r}}^T$.
$\Pi$ denotes degrees of freedom on the assembled subdomain vertex space, while $r$ denotes degrees of freedom on the broken space given by the subdomain interior and duplicated portions of the subdomain interface.
\label{def:bddcpreconditioner}
\end{definition}

% -- Injection -----------------------------------------------------------------
\subsection{Injection Operators}
\input 05-DomainDecomposition/01-01-injection

% -- Fast Diagonalization ------------------------------------------------------
\subsection{Subdomain Solver with Fast Diagonalization}
\input 05-DomainDecomposition/01-02-fastdiagonalization
