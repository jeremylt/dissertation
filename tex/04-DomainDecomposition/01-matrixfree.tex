For high-order matrix-free BDDC, we decompose the domain $\Omega$ into a series of subdomains $\Omega^i$ given by the individual elements.
With BDDC, we utilize four spaces on each subdomain.
The interface between subdomains is given by $\Gamma = \bigcup \partial \Omega^i \backslash \partial \Omega$ and the interface of subdomain $\Omega^i$ is given by $\Gamma^i = \delta \Omega^i \bigcup \Gamma$.
The interior of each subdomain, $\text{I}$, is given by the remaining degrees of freedom on the element.

Each subdomain problem can therefore be written as
\begin{equation}
{\color{burgundy}\mathbf{A}}^e =
\left( \begin{array}{c c}
{\color{burgundy}\mathbf{A}}_{\text{I}, \text{I}}^e  &  {\color{burgundy}\mathbf{A}}_{\Gamma, \text{I}}^{e, T}  \\
{\color{burgundy}\mathbf{A}}_{\Gamma, \text{I}}^e    &  {\color{burgundy}\mathbf{A}}_{\Gamma, \Gamma}^e         \\
\end{array} \right)
\end{equation}
where ${\color{burgundy}\mathbf{A}}^e = {\color{blue(ncs)}\mathbf{B}}^T {\color{applegreen}\mathbf{D}} {\color{blue(ncs)}\mathbf{B}}$, as shown in Equation \ref{eq:localoperator}.

This subdomain problem is assembled into the global problem in the typical finite element approach given by Equation \ref{eq:libceed_representation}, but in BDDC we create a subassembled problem that is easier to invert than the global problem by duplicating broken degrees of freedom along the subdomain boundaries.
Only the corner, or vertex, degrees of freedom from each element are assembled into a global coarse problem, and the solutions on the global coarse problem and broken subdomain problems are assembled into an approximate solution to the global problem.

The coarse grid subdomain nodes $\Pi^i$ are given by the corners, so we have $4$ coarse grid subdomain nodes for each element in two dimensions and $8$ coarse grid subdomain nodes in three dimensions.
The remainder of the subdomain we denote with $\text{r}$.
The subassembled problem is therefore given by
\begin{equation}
\hat{\color{burgundy}\mathbf{A}} = \sum_{e = 1}^N \mathbf{R}^{e, T} \hat{\color{burgundy}\mathbf{A}}^e \mathbf{R}^e
\label{eq:subassembled}
\end{equation}
where
\begin{equation}
\hat{\color{burgundy}\mathbf{A}}^e =
\left( \begin{array}{c c}
{\color{burgundy}\mathbf{A}}_{\text{r}, \text{r}}^e  &  {\color{burgundy}\mathbf{A}}_{\Pi, \text{r}}^{e, T}  \\
{\color{burgundy}\mathbf{A}}_{\Pi, \text{r}}^e       &  {\color{burgundy}\mathbf{A}}_{\Pi, \Pi}^e            \\
\end{array} \right).
\end{equation}
In this formulation, the broken degrees of freedom found in $\Gamma^i - \Pi^i$ are duplicated for each subdomain in contrast to overlapping domain decomposition methods, where only one degree of freedom would be used at these locations.

% -- Injection -----------------------------------------------------------------
\subsection{Injection Operators}
\input 04-DomainDecomposition/01-01-injection

% -- Saddle Point Problem ------------------------------------------------------
\subsection{Saddle Point Problem}
\input 04-DomainDecomposition/01-02-saddle

% -- Fast Diagonalization ------------------------------------------------------
\subsection{Subdomain Solver with Fast Diagonalization}
\input 04-DomainDecomposition/01-03-fastdiagonalization
