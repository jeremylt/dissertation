High-order finite elements offer high accuracy and exponential convergence for sufficiently smooth problems.
In general, the discretization error in the finite element approximation is given by
\begin{equation}
E_{H^1 \left( \Omega \right)} \leq C \left( p \right) h^{\text{min} \left( k, p + 1 \right)}
\label{eq:spectral_convergence}
\end{equation}
where $p$ is the polynomial order of the mesh, $h$ is the size of the finite elements, and $k$ is the order of the Sobolev space to which the true solution belongs.
The coefficient $C$ does depend upon the polynomial order of the finite element basis, but in practice the exponential term dominates this error approximation.

Often exponential convergence is not required to meet engineering tolerances and the smoothness of the solution may prevent exponential convergence from being achieved in practical problems.
However, high-order finite elements will still offer convergence that is no worse than the convergence on a comparable low-order mesh with a larger number of elements.
For these problems, high-order finite elements implemented in a matrix-free fashion still offer the storage and FLOPs benefits detailed above in Section \ref{sec:storageandflops}.
For further discussion of the convergence of high-order methods, see \cite{babuvska1994p,babuska1982rates,guo1986hp}, among others.

Demkowicz, Oden, and Rachowicz, et al. discussed $hp$ adaptivity in the context of minimizing the total number of degrees of freedom (DoFs) required to achieve a target accuracy \cite{demkowicz1989toward, oden1989toward, rachowicz1989toward}.
However, in $hp$ adaptivity, the polynomial order of each finite element may differ.
While problems utilizing these types of discretizations can be implemented in a matrix-free fashion with the preconditioners discussed in Chapter \ref{ch:MultigridMethods} and Chapter \ref{ch:DomainDecomposition}, we instead focus on meshes where all elements have the same polynomial order as to simplify the analysis and implementation details.
For discussion of implementation details for $hp$ finite element methods, see Frauenfelder and Lage's discussion of Concepts \cite{frauenfelder2002concepts} and Bangerth and Kayser-Herold's discussion of deal.II \cite{bangerth2009data}.