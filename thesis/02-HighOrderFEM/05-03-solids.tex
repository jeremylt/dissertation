The solid mechanics mini-application solves the steady-state static momentum balance equations using unstructured high-order finite element spatial discretizations.
As with the fluid dynamics mini-application, the solid mechanics elasticity mini-application has been developed using PETSc with libCEED so that the matrix-free operator application is separated from the parallelization, meshing, and solver concerns.
This mini-application uses $p$-multigrid preconditioning as described in Chapter \ref{ch:MultigridMethods}.

In this mini-application, we consider three formulations used in solid mechanics applications: linear elasticity, Neo-Hookean hyperelasticity at small strain, and Neo-Hookean hyperelasticity at finite strain.
We provide the strong and weak forms of the static balance of linear momentum in the small strain and finite strain regimes below.
The stress-strain relationship for each of the material models is provided.
Due to the nonlinearity of material models in Neo-Hookean hyperelasticity, the Newton linearization of the material models is also provided.

Linear elasticity and small-strain hyperelasticity can both be obtained from the finite-strain hyperelastic formulation by linearization of geometric and constitutive nonlinearities.
The effect of these linearizations is sketched in Figure \ref{fig:hyperelastic-cd}, where $\boldsymbol \sigma$ and $\boldsymbol \epsilon$ are stress and strain, respectively, in the small strain regime, while $\mathbf S$ and $\mathbf E$ are their finite-strain generalizations, the second Piola-Kirchoff tensor and Green-Lagrange strain tensor, respectively, defined in the initial configuration, and $\mathsf C$ is a linearized constitutive model.

\begin{figure}
$$
      \begin{CD}
        {\overbrace{\mathbf S \left( \mathbf E \right)}^{\text{Finite Strain Hyperelastic}}}
        @>{\text{constitutive}}>{\text{linearization}}>
        {\overbrace{\mathbf S = \mathsf C \mathbf E}^{\text{St. Venant-Kirchoff}}} \\
        @V{\text{geometric}}V{\begin{smallmatrix}\mathbf E \to \boldsymbol \epsilon \\ \mathbf S \to \boldsymbol \sigma \end{smallmatrix}}V
        @V{\begin{smallmatrix}\mathbf E \to \boldsymbol \epsilon \\ \mathbf S \to \boldsymbol \sigma \end{smallmatrix}}V{\text{geometric}}V \\
        {\underbrace{\boldsymbol \sigma \left( \boldsymbol \epsilon \right)}_\text{Small Strain Hyperelastic}}
        @>{\text{constitutive}}>\text{linearization}>
        {\underbrace{\boldsymbol \sigma = \mathsf C \boldsymbol \epsilon}_\text{Linear Elastic}}
      \end{CD}
$$
\caption{Linearization of Hyperelasticitic Model}
\label{fig:hyperelastic-cd}
\end{figure}

\subsubsection{Hyperelasticity at Finite Strain}

In the total Lagrangian approach for the Neo-Hookean hyperelasticity problem, the discrete equations are formulated with respect to the initial configuration.
In this formulation, we solve for displacement $\mathbf u \left( \mathbf X \right)$ in the initial frame $\mathbf X$.
The notation for elasticity at finite strain is inspired by \cite{holzapfel2000nonlinear} to distinguish between the current and initial configurations.
We denote the initial frame by capital letters and the current frame by lower case letters.

The strong form of the static balance of linear-momentum at finite strain is given by
\begin{equation}
   - \nabla_X \cdot \mathbf{P} - \rho_0 \mathbf{g} = \mathbf{0}
   \label{eq:sblFinS}
\end{equation} 
where the $\nabla_X$ indicates that the gradient is calculated with respect to the initial configuration in the finite strain regime.
$\mathbf{P}$ is the first Piola-Kirchhoff stress tensor and $\mathbf{g}$ is the prescribed forcing function, while $\rho_0$ gives the initial mass density.
The tensor $\mathbf{P}$ is not symmetric, living in the current configuration on the left and the initial configuration on the right.

The first Piola-Kirchoff stress tensor can be decomposed as
\begin{equation}
   \mathbf{P} = \mathbf{F} \, \mathbf{S},
   \label{eq:1st2nd}
\end{equation}
where $\mathbf{S}$ is the second Piola-Kirchhoff stress tensor, a symmetric tensor defined entirely in the initial configuration, and $\mathbf{F} = \mathbf I_3 + \nabla_X \mathbf u$ is the deformation gradient.
Different constitutive models can define $\mathbf{S}$.

For the constitutive modeling of hyperelasticity at finite strain, we begin by defining two symmetric tensors in the initial configuration, the right Cauchy-Green tensor
\begin{equation}
\mathbf C = \mathbf F^T \mathbf F
\end{equation}
and the Green-Lagrange strain tensor
\begin{equation}
   \mathbf E = \frac 1 2 \left( \mathbf C - \mathbf I_3 \right) = \frac 1 2 \left( \nabla_X \mathbf u + \left( \nabla_X \mathbf u \right)^T + \left( \nabla_X \mathbf u \right)^T \nabla_X \mathbf u \right).
   \label{eq:green-lagrange-strain}
\end{equation}
The Green-Lagrange strain tensor converges to the linear strain tensor $\boldsymbol \epsilon$ in the small-deformation limit.
The constitutive models we consider express $\mathbf S$ as a function of $\mathbf E$ and are appropriate for large deformations.

In their most general form, constitutive models define $\mathbf S$ in terms of state variables.
For the model used in the present mini-application, the state variables are constituted by the vector displacement field $\mathbf u$ and its gradient $\nabla \mathbf u$.

The constitutive model $\mathbf S \left( \mathbf E \right)$ is a tensor-valued function of a tensor-valued input.
An arbitrary choice of such a function will generally not be invariant under orthogonal transformations and thus will not admissible as a physical model must not depend on the coordinate system chosen to express it.
In particular, given an orthogonal transformation $Q$, we require
\begin{equation}
   Q \mathbf S \left( \mathbf E \right) Q^T = \mathbf S \left( Q \mathbf E Q^T \right),
   \label{eq:elastic-invariance}
\end{equation}
which means that we can change our initial frame before or after computing $\mathbf S$ and get the same result.
Materials with constitutive relations in which $\mathbf S$ is uniquely determined by $\mathbf E$ while satisfying the invariance property given by Equation \ref{eq:elastic-invariance} are known as Cauchy elastic materials.

We define a strain energy density functional $\Phi \left( \mathbf E \right) \in \mathbb{R}$ and obtain the strain energy from its gradient,
\begin{equation}
\mathbf S \left( \mathbf E \right) = \frac{\partial \Phi}{\partial \mathbf E}.
\label{eq:strain-energy-grad}
\end{equation}
The strain energy density functional can only depend upon invariants, which are scalar-valued functions $\gamma$ satisfying
\begin{equation}
\gamma \left( \mathbf E \right) = \gamma \left( Q \mathbf{E} Q^T \right)
\end{equation}
for all orthogonal matrices $Q$.

We will assume without loss of generality that $\mathbf E$ is diagonal and take its set of eigenvalues as the invariants.
It is immediately clear that there can be only three invariants and that there are many alternate choices, such as $\operatorname{trace}(\mathbf E), \operatorname{trace}(\mathbf E^2), \lvert \mathbf E \rvert$, and combinations thereof.
It is common in the literature for invariants to be taken from $\mathbf C = \mathbf I_3 + 2 \mathbf E$ instead of $\mathbf E$.

We use the compressible Neo-Hookean model,
\begin{equation}
   \begin{aligned}
   \Phi \left( \mathbf E \right) &= \frac{\lambda}{2} \left( \log J \right)^2 + \frac \mu 2 \left(\operatorname{trace} \mathbf C - 3 \right) - \mu \log J \\
     &= \frac{\lambda}{2}\left( \log J \right)^2 + \mu \operatorname{trace} \mathbf E - \mu \log J,
   \end{aligned}
   \label{eq:neo-hookean-energy}
\end{equation}
where $J = \lvert \mathbf F \rvert = \sqrt{\lvert \mathbf C \rvert}$ is the determinant of deformation, volumetric change.
$\lambda$ and $\mu$ are the Lamé parameters in the infinitesimal strain limit, given by
\begin{equation}
\lambda = \frac{E \nu}{\left( 1 + \nu \right) \left( 1 - 2 \nu \right)}, \hspace{5mm}
\mu = \frac{E}{2 \left( 1 + \nu \right)},
\end{equation}
where $E$ is the Young's modulus and $\nu$ is the Poisson's ratio for the materiel.

To evaluate Equation \ref{eq:strain-energy-grad}, we make use of
\begin{equation}
   \frac{\partial J}{\partial \mathbf E} = \frac{\partial \sqrt{\lvert \mathbf C \rvert}}{\partial \mathbf E} = \lvert \mathbf C \rvert^{-1/2} \lvert \mathbf C \rvert \mathbf C^{-1} = J \mathbf C^{-1},
\end{equation}
where the factor of $\frac 1 2$ has been absorbed due to the fact that $\mathbf C = \mathbf I_3 + 2 \mathbf E$.
Carrying through the differentiation in Equation \ref{eq:strain-energy-grad} for the model in Equation \ref{eq:neo-hookean-energy}, we arrive at
\begin{equation}
   \mathbf S = \lambda \log J \mathbf C^{-1} + \mu (\mathbf I_3 - \mathbf C^{-1})
   \label{eq:neo-hookean-stress}
\end{equation}
which is the strain energy for Neo-Hookean hyperelasticity at finite strain.

\subsubsection{Hyperelasticity Weak Form}

We multiply Equation \ref{eq:sblFinS} by a test function $\mathbf v$ and integrate by parts to obtain the weak form for finite-strain hyperelasticity:
find $\mathbf u \in \mathcal V \subset H^1 \left( \Omega_0 \right)$ such that
\begin{equation}
    \int_{\Omega_0}{\nabla_X \mathbf{v} \!:\! \mathbf{P}} \, dV
    - \int_{\Omega_0}{\mathbf{v} \cdot \rho_0 \mathbf{g}} \, dV
    - \int_{\partial \Omega_0}{\mathbf{v} \cdot (\mathbf{P} \cdot \hat{\mathbf{N}})} \, dS
    = 0, \quad \forall \mathbf v \in \mathcal V,
   \label{eq:hyperelastic-weak-form-initial}
\end{equation}    
where $\mathbf{P} \cdot \hat{\mathbf{N}}|_{\partial\Omega}$ is replaced by any prescribed force/traction boundary condition written in terms of the initial configuration.

This equation contains material and constitutive nonlinearities in defining $\mathbf S \left( \mathbf E \right)$, as well as geometric nonlinearities through $\mathbf P = \mathbf F\, \mathbf S$, $\mathbf E \left( \mathbf F \right)$, and the body force $\mathbf g$, which must be pulled back from the current configuration to the initial configuration.
Discretization of Equation \ref{eq:hyperelastic-weak-form-initial} produces a finite-dimensional system of nonlinear algebraic equations, which we solve using Newton-Raphson methods.
One attractive feature of Galerkin discretization is that we can arrive at the same linear system by discretizing the Newton linearization of the continuous form; that is, discretization and differentiation (Newton linearization) commute.

\subsubsection{Newton Linearization}

To derive a Newton linearization of Equation \ref{eq:hyperelastic-weak-form-initial}, we begin by expressing the derivative of Equation \ref{eq:1st2nd} in incremental form,
\begin{equation}
   \text{d} \mathbf P = \frac{\partial \mathbf P}{\partial \mathbf F} \!:\! \text{d} \mathbf F = \text{d} \mathbf F\, \mathbf S + \mathbf F \underbrace{\frac{\partial \mathbf S}{\partial \mathbf E} \!:\! \text{d} \mathbf E}_{\text{d} \mathbf S}
   \label{eq:diff-P}
\end{equation}
where
\begin{equation}
   \text{d} \mathbf E = \frac{\partial \mathbf E}{\partial \mathbf F} \!:\! \text{d} \mathbf F = \frac 1 2 \left( \text{d} \mathbf F^T \mathbf F + \mathbf F^T \text{d} \mathbf F \right)
\end{equation}
and $\text{d}\mathbf F = \nabla_X\text{d}\mathbf u$.
The quantity ${\partial \mathbf S} / {\partial \mathbf E}$ is known as the incremental elasticity tensor.
We now evaluate $\text{d} \mathbf S$ for the Neo-Hookean model given in Equation \ref{eq:neo-hookean-stress},
\begin{equation}
   \text{d}\mathbf S = \frac{\partial \mathbf S}{\partial \mathbf E} \!:\! \text{d} \mathbf E
   = \lambda \left(\mathbf C^{-1} \!:\! \text{d}\mathbf E \right) \mathbf C^{-1}
     + 2 \left(\mu - \lambda \log J \right) \mathbf C^{-1} \text{d}\mathbf E \, \mathbf C^{-1},
   \label{eq-neo-hookean-incremental-stress}
\end{equation}
where we have used
\begin{equation}
   \text{d} \mathbf C^{-1} = \frac{\partial \mathbf C^{-1}}{\partial \mathbf E} \!:\! \text{d}\mathbf E
   = -2 \mathbf C^{-1} \text{d} \mathbf E \, \mathbf C^{-1} .
\end{equation}

\subsubsection{St. Venant-Kirchoff}

One can linearize Equation \ref{eq:neo-hookean-stress} around $\mathbf E = 0$, for which $\mathbf C = \mathbf I_3 + 2 \mathbf E \to \mathbf I_3$ and $J \to 1 + \operatorname{trace} \mathbf E$, therefore Equation \ref{eq:neo-hookean-stress} reduces to
\begin{equation}
      \mathbf S = \lambda (\operatorname{trace} \mathbf E) \mathbf I_3 + 2 \mu \mathbf E,
      \label{eq:st-venant-kirchoff}
\end{equation}
which is the St. Venant-Kirchoff model.
This model has constitutive linearization without geometric linearization, as mentioned in Figure \ref{fig:hyperelastic-cd}.

This model can be used for geometrically nonlinear mechanics such as snap-through of thin structures, but it is inappropriate for large strain.

\subsubsection{Hyperelasticity at Small Strain}

Alternatively, one can drop the geometric nonlinearities, $\mathbf E \to \boldsymbol \epsilon$ and $\mathbf C \to \mathbf I_3$, while retaining the nonlinear dependence on $J \to 1 + \operatorname{trace} \boldsymbol \epsilon$, thereby yielding the Neo-Hookean hyperelasticity at small strain.

In this case, the strain energy density function is given by
\begin{equation}
\Phi \left( \boldsymbol \epsilon \right) = \lambda \left( 1 + \operatorname{trace} \boldsymbol \epsilon \right) \left( \log \left( 1 + \operatorname{trace} \boldsymbol \epsilon \right) - 1 \right) + \mu \boldsymbol \epsilon : \boldsymbol \epsilon
\end{equation}
and the corresponding constitutive law is given by
\begin{equation}
\boldsymbol \sigma = \lambda \log \left( 1 + \operatorname{trace} \boldsymbol \epsilon \right) \mathbf I_3 + 2 \mu \boldsymbol \epsilon.
\end{equation}

\subsubsection{Linear Elasticity}

The linear elasticity model can be derived by linearizing both the geometric and constitutive nonlinearities in the finite strain model, as shown in Figure \ref{fig:hyperelastic-cd}, or independently derived from the static balance of linear momentum.

The strong form of the static balance of linear momentum at small strain for the three dimensional linear elasticity problem is given by \cite{hughes2012finite} as
\begin{equation}
\nabla \cdot \boldsymbol{\sigma} + \boldsymbol{g} = \boldsymbol{0}
\end{equation}
where $\boldsymbol{\sigma}$ is the stress function and $\boldsymbol{g}$ is the forcing function.
This strong form has the corresponding weak form
\begin{equation}
\int_{\Omega} \nabla \mathbf{v} : \boldsymbol{\sigma} dV - \int_{\partial \Omega} \mathbf{v} \cdot \left( \boldsymbol{\sigma} \cdot \hat{\mathbf{n}} \right) dS - \int_{\Omega} \mathbf{v} \cdot \mathbf{g} dV = 0, \forall \mathbf{v} \in \mathcal{V}
\end{equation}
for some displacement $\mathbf{u} \in \mathcal{V} \subset H^1 \left( \Omega \right)$, where $:$ denotes contraction over both components and dimensions.

In the linear elasticity constitutive model, the symmetric strain tensor is given by
\begin{equation}
\boldsymbol{\epsilon} = \frac{1}{2} \left( \nabla \mathbf{u} + \nabla \mathbf{u}^T \right)
\end{equation}
and the linear elasticity constitutive law is given by $\boldsymbol{\sigma} = \mathsf{C} : \boldsymbol{\epsilon}$ where
\begin{equation}
\mathsf{C} =
\begin{bmatrix}
   \lambda + 2\mu & \lambda & \lambda & & & \\
   \lambda & \lambda + 2\mu & \lambda & & & \\
   \lambda & \lambda & \lambda + 2\mu & & & \\
   & & & \mu & & \\
   & & & & \mu & \\
   & & & & & \mu
\end{bmatrix}.
\end{equation}

\subsubsection{Ongoing Research}

\begin{figure}[ht!]
\includegraphics[width=.99\linewidth]{../img/SolidTwistExample}
\caption{Strain Energy Density in Twisted Neo-Hookean Beam}
\label{fig:solidtwist}
\end{figure}

Figure \ref{fig:solidtwist} shows the strain energy density for a beam undergoing a twist with Neo-Hookean hyperelastic modeling at finite strain.
This mini-application can run on host or device processors.
Areas of ongoing research include mixed finite element formulations for the displacement and pressure spaces, efficient load continuation techniques, and preconditioning improvements.
For further information about the libCEED solid mechanics mini-application, see \cite{imece2020} and \cite{mehraban2021simulating}; we summarize the key findings below.

\subsubsection{Nearly Incompressible Linear Elasticity}

In this section, we provide the results of performance studies for the linear elasticity formulation on a three dimensional unit cube using a manufactured solution for a range of Poisson's ratios that approach the incompressible limit of $\nu = 0.5$.
The manufactured solution is reproducing a final displacement given by
\begin{equation}
\begin{pmatrix}
u_1 \\
u_2 \\
u_3 \\
\end{pmatrix} =
\begin{pmatrix}
e^{2 x} \sin \left( 3 y \right) \cos \left( 4 z \right) \\
e^{3 x} \sin \left( 4 y \right) \cos \left( 2 z \right) \\
e^{4 x} \sin \left( 2 y \right) \cos \left( 3 z \right) \\
\end{pmatrix}
\end{equation}
with full Dirichlet boundary conditions on the outside of the box.

The performance study was conducted on a two-socket AMD EPYC 7452 machine.
Each socket contains 32 CPU cores with a base clock speed of 2.35 GHz and 128 MB of L3 cache.
The NPS4 BIOS configuration was used and processes were bound to cores.
MPICH-3.3.2 was used with PETSc \cite{petsc-user-ref} version 3.14 and libCEED \cite{libceed} version 0.7.

In Figure \ref{fig:error-time} and Figure \ref{fig:error-cost} we investigate Pareto optimal configurations.
A point is Pareto optimal if we cannot decrease error, along the $y$ axis, without increasing cost in terms time, along the $x$ axis.
Note that as we approach the incompressible limit, we omit linear elements to avoid locking.

\begin{figure}[pbt!]
 \begin{center}
      \includegraphics[width=1\textwidth]{../img/error-time.pdf}
\end{center}
\caption{Error vs Time for $\nu = 0.3$, $\nu = 0.49$, $\nu = 0.49999$ and $\nu = 0.499999$ for Linear Elasticity}
    \label{fig:error-time}
\end{figure}

In Figure \ref{fig:error-time}, the Pareto optimal configurations are toward the lower left of each pane, with $p = 3$ and $p = 4$ giving the fastest solutions for any error tolerance, while low-order elements ($p = 1$ and $p = 2$) are increasingly further from the Pareto front for larger values of $\nu$.
Each horizontal series of the same color represents strong scaling of a given resolution $h$ and $p$.

\begin{figure}[pbt!]
 \begin{center}
      \includegraphics[width=1\textwidth]{../img/error-cost.pdf}
\end{center}
\caption{Error vs Cost for $\nu = 0.3$, $\nu = 0.49$, $\nu = 0.49999$ and $\nu = 0.499999$ for Linear Elasticity}
    \label{fig:error-cost}
\end{figure}

In Figure \ref{fig:error-cost}, the Pareto optimal configurations are toward the lower left of each pane indicate higher order $p$ is most cost-efficient.
Each horizontal series of the same color represents a strong scaling study at fixed $h$ and $p$, with perfect strong scaling manifesting when all the dots are collocated.
The more expensive models tend to exhibit better strong scaling because they have more work over which to amortize the inherent communication costs, while small models are much more cost-efficient to run on a single core.

In addition, we conducted an $h$-refinement study for polynomials of order 1 through 4 for the compressible case with $\nu = 0.3$ to determine the global rate of convergence of our implementation as the grid size is decreased.
Figure \ref{fig:lin-elas-conv} is a \texttt{log-log} plot of $h$ versus $L^2$ error that represents the convergence of our implementation using grid sizes $h = 1/3$ to $h = 1/80$ for $\nu = 0.3$.

\begin{figure}[hbt!]
 \begin{center}
      \includegraphics[width=1\textwidth]{../img/conv.pdf}
\end{center}
\caption{\texttt{log-log} Plot of $L^2$ Error vs $h$ for Polynomial Orders 1-4 with $\nu = 0.3$ for Hyperelasticity at Finite Strain}
    \label{fig:lin-elas-conv}
\end{figure}

In Figure \ref{fig:lin-elas-conv}, for each polynomial order, we calculate the slope of the line that is the best fit in the least square sense.
The slopes of the lines of best fit in the least square sense in Figure \ref{fig:lin-elas-conv} are 2.07, 4.00, 4.45, 4.75 for polynomials of order 1 through 4 respectively.
We notice stagnation of the solver for $h = 1/40$ and $h = 1/80$ respectively with $p = 3$ and $p = 4$ due to the default tolerances that we used for the iterative solver.
For polynomial of order 1, we notice accelerated convergence behavior for coarse meshes, therefore they are not considered in the computation of the slope.
We observed the expected spectral convergence; fixing the grid size constant while increasing the polynomial order provides the fastest time to converge to the desired solution given a target error tolerance.

\subsubsection{Neo-Hookean Hyperelasticity at Finite Strain}

In this section, we provide results for a performance study with Neo-Hookean hyperelasticity at finite strain.

\begin{figure}[ht!]
 \begin{center}
      \includegraphics[width=0.45\textwidth]{../img/cylinder.pdf}
\end{center}
\caption{Hexahedral Mesh for Cylindrical Tube Bending Problem}
    \label{fig:hyper-cylinder}
\end{figure}

For this test case, we simulate an assistive finger device shown in Figure \ref{fig:hyper-cylinder}.
The length of the tube is 100mm, with circular cross-section with inner diameter 10mm and outer diameter 15mm.
On the left end of the device we apply a zero Dirichlet boundary condition and on the right we apply a non-homogeneous Dirichlet boundary condition of 50mm in the negative $y$ direction, downward, simulating a tube bending problem.
We use a Poisson's ration of $\nu = 0.49$ and Young's Modulus of $E = 1.0$ MPa.

The performance study was conducted on a machine with a single AMD RYZEN 7 3700X 8-core 3.6GHz (4.4 GHz Max Boost) Socket AM4 65W 100-1000000711BOX Desktop Processor with G.SKILL Ripjaws V Series 32GB (2 x 16GB) 288-Pin DDR4 SDRAM DDR4 3200 (PC4 25600) Desktop Memory Model F4-3200C16D-32GV.
PETSc \cite{petsc-user-ref} version 3.13 was used with libCEED \cite{libceed} version 0.6.

\begin{figure}[ht!]
\begin{center}
\includegraphics[angle=0,width=0.55\linewidth]{../img/tube27.png}
\end{center}
\caption{Displacement Magnitude in mm for Deformed Mesh}
\label{fig:hyper-press} 
\end{figure}

Figure \ref{fig:hyper-press} shows the final displacement of the tube in the simulation.

\begin{table}[ht!]
\begin{center}
\begin{tabular}{c c c c c c} 
 \toprule
 Elem & Nodes & dof & Load Increments & Time & np \\ [0.5ex] 
 \midrule
736   &  7,434 &  22,302 & 500  &  474.05   & 8\\ 
5,550 & 52,920 & 158,760 & 750  &  8,749.68 & 8\\
 \bottomrule
\end{tabular}
\end{center}
\caption{Total CPU Time with $\nu = 0.49$ for Hyperelasticity at Finite Strain}
 \label{table:hyper-meshSizes}
\end{table}

In Table \ref{table:hyper-meshSizes}, we see the the total CPU item required to complete the simulation.
This simulation is currently performing poorly in the nearly incompressible regime, due to the large number of load increments required.
Improved load increment strategies are an area of ongoing research for this mini-application.