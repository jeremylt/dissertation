Consider a scalar Toeplitz operator $L_h$ on an infinite one dimensional uniform grid $G_h$,
\begin{equation}
\begin{split}
L_h \mathrel{\hat{=}} \left[ s_\kappa \right]_h \left( \kappa \in V \right)\\
L_h w_h \left( x \right) = \sum_{\kappa \in V} s_\kappa w_h \left( x + \kappa h \right)
\end{split}
\end{equation}
where $V \subset \mathbb{Z}$ is a finite index set, $s_\kappa \in \mathbb{R}$ are constant coefficients, and $w_h \left( x \right)$ is a $l^2$ function on $G_h$.
In terms of stencils, $s_k$ are stencil weights that are nonzero on the neighborhood $\kappa \in V$.

Since $L_h$ is Toeplitz, it can be diagonalized by the standard Fourier modes $\varphi \left( \theta, x \right) = e^{\imath \theta x / h}$.
These Fourier modes alias according to $\varphi \left( \theta + 2 \pi / h, x \right)$ on all grid points $x \in h \mathbb{Z}$.

\begin{definition}[Symbol of $L_h$]\label{def:symbol}
If for all grid functions $\varphi \left( \theta, x \right)$ we have
\begin{equation}
L_h \varphi \left( \theta, x \right) = \tilde{L}_h \left( \theta \right) \varphi \left( \theta, x \right)
\end{equation}
then we define $\tilde{L}_h \left( \theta \right) = \sum_{\kappa \in V} s_\kappa e^{\imath \theta \kappa}$ as the symbol of $L_h$, where $\imath^2 = -1$.
\end{definition}

This definition can be extended to a $q \times q$ linear system of operators by
\begin{equation}
\mathbf{L}_h =
\begin{bmatrix}
    L_h^{1, 1} && \cdots && L_h^{1, q}        \\
    \vdots               && \vdots && \vdots  \\
    L_h^{q, 1} && \cdots && L_h^{q, q}        \\
\end{bmatrix}
\end{equation}
where $L_h^{i, j}$, $i, j \in \left\lbrace 1, 2, \dots, q \right\rbrace$ is given by scalar Toeplitz operators describing how component $j$ appears in the equation for component $i$.
The symbol of $\mathbf{L}_h$, denoted $\tilde{\mathbf{L}}_h$, is a $q \times q$ matrix valued function of $\theta$ given by $\left[ \tilde{\mathbf{L}}_h \right]_{i, j} = \tilde{L}_h^{i, j} \left( \theta \right)$.
Note that for a system of equations representing an error propagation operator in a relaxation scheme, the spectral radius of the symbol matrix determines now rapidly the scheme decreases error at a target frequency.

These definitions are extended to $d$ dimensions by taking the neighborhood $V \subset \mathbb{Z}^d$ and letting $\boldsymbol{\theta} \in \left[ - \pi, \pi \right)^d$.
For standard coarsening in the analysis of h-multigrid, low frequencies are given by $\boldsymbol{\theta} \in T^{\text{low}} = \left[ - \pi / 2, \pi / 2 \right)^d$ and high frequencies are given by $\boldsymbol{\theta} \in \left[ - \pi, \pi \right)^d \ T^{\text{low}}$, or equivalently via periodicity, $\theta \in T^{\text{high}} = \left[ - \pi / 2, 3 \pi / 2 \right)^d \setminus T^{\text{low}}$.

We can compute the symbol of a $p \times p$ linear system of operators representing a discretized PDE.
We start with a system of equations representing a Galerkin operator, such as in Equation \ref{eq:jacobian_form}, but we omit boundary terms in this derivation, as they are not present on the infinite uniform grid $G_h$.

Using the algebraic representation of PDE operators given in Chapter \ref{ch:HighOrderFEM}, the PDE operator ${\color{burgundy}\mathbf{A}}$ is of the form
\begin{equation}
\begin{tabular}{c}
${\color{burgundy}\mathbf{A}} = \mathbf{P}^T \mathbf{G}^T {\color{burgundy}\mathbf{A}}^e \mathbf{G} \mathbf{P}$\\
${\color{burgundy}\mathbf{A}}^e = {\color{blue(ncs)}\mathbf{B}}^T {\color{applegreen}\mathbf{D}} {\color{blue(ncs)}\mathbf{B}}$
\end{tabular}
\label{eq:localoperator}
\end{equation}
where $\mathbf{P}$ and $\mathbf{G}$ represent the element assembly operators, ${\color{blue(ncs)}\mathbf{B}}$ is a basis operator which computes the values and derivatives of the basis functions at the quadrature points, and ${\color{applegreen}\mathbf{D}}$ is a block diagonal operator which provides the pointwise application of the bilinear form on the quadrature points, to include quadrature weights and the change in coordinates between the physical and reference space.

We focus on LFA of the element operator, ${\color{burgundy}\mathbf{A}}^e$, on the mesh grid with points given by $x_i$, for $i \in \left\lbrace 1, 2, \dots, p + 1 \right\rbrace$, where $p$ is the polynomial degree of the basis.
We first develop the LFA in one dimension for scalar operators and then extend this analysis to more general operators.

In one dimension, the nodes on the left and right boundaries of the element map to the same Fourier mode when localized to nodes unique to a single finite element, so we can compute the symbol matrix as
\begin{equation}
\tilde{{\color{burgundy}\mathbf{A}}} = \mathbf{Q}^T \left( {\color{burgundy}\mathbf{A}}^e \odot \left[ e^{\imath \left( x_j - x_i \right) \theta / h} \right] \right) \mathbf{Q}
\label{eq:elementsymbol1d}
\end{equation}
where $\odot$ represents pointwise multiplication of the elements, $h$ is the length of the element, and $i, j \in \left\lbrace 1, 2, \dots, p + 1 \right\rbrace$.
In the pointwise product ${\color{burgundy}\mathbf{A}}^e \odot \left[ e^{\imath \left( x_j - x_i \right) \theta / h} \right]$, the $\left( i, j \right)$ entry is given by $\left[ {\color{burgundy}\mathbf{A}}^e \right]_{i, j} e^{\imath \left( x_j - x_i \right) \theta / h}$.
$\mathbf{Q}$ is a $\left( p + 1 \right) \times p$ matrix that localizes Fourier modes to each element, given by
\begin{equation}
\mathbf{Q} =
\begin{bmatrix}
    \mathbf{I}   \\
    \mathbf{e}_0 \\
\end{bmatrix} =
\begin{bmatrix}
    1      && 0      && \cdots && 0      \\
    0      && 1      && \cdots && 0      \\
    \vdots && \vdots && \vdots && \vdots \\
    0      && 0      && \cdots && 1      \\
    1      && 0      && \cdots && 0      \\
\end{bmatrix}.
\label{eq:fouriermodelocalization1d}
\end{equation}

The computation of this symbol matrix extends to more complex PDE with multiple components and in higher dimensions.

Multiple components are supported by extending the $p \times p$ system of Toeplitz operators in Equation \ref{eq:elementsymbol1d} to a $\left( n \cdot p \right) \times \left( n \cdot p \right)$ system of operators, where $n$ is the number of components in the PDE.
The localization operator $\mathbf{Q}$ for a multi-component PDE is given by $\mathbf{Q}_n = \mathbf{I}_n \otimes \mathbf{Q}$.
In general, we omit the subscript indicating the number of components for the localization operator.

The infinite uniform grid $G_h$ is extended into higher dimensions by taking the direct sum of the one dimensional grid.
In a similar fashion to how tensor products are used to extend the one dimensional bases into higher dimensions, the localization of Fourier modes in two dimensions is given by
\begin{equation}
\mathbf{Q}_{\text{2d}} = \mathbf{Q} \otimes \mathbf{Q}
\end{equation}
and the localization in three dimensions is given by
\begin{equation}
\mathbf{Q}_{\text{3d}} = \mathbf{Q} \otimes \mathbf{Q} \otimes \mathbf{Q}.
\end{equation}
Again, we generally omit the subscript indicating the dimension of the Fourier mode localization operator.

\begin{definition}[Symbol of High-Order Finite Element Operators]
The symbol matrix of a finite element operator for an arbitrary second-order PDE with any number of components, basis order, and dimension is given by
\begin{equation}
\tilde{{\color{burgundy}\mathbf{A}}} \left( \boldsymbol{\theta} \right) = \mathbf{Q}^T \left( {\color{burgundy}\mathbf{A}}^e \odot \left[ e^{\imath \left( \mathbf{x}_j - \mathbf{x}_i \right) \cdot \boldsymbol{\theta} / \mathbf{h}} \right] \right) \mathbf{Q}
\label{eq:symbolhighorder}
\end{equation}
where $\odot$ represents pointwise multiplication of the elements, $\mathbf{h}$ is the length of the element in each dimension, $\boldsymbol{\theta}$ is the target frequency in each dimension, $i, j \in \left\lbrace 1, 2, \dots, n \cdot \left( p + 1 \right)^d \right\rbrace$, $n$ is the number of components, $p$ is the polynomial degree of the discretization, and $d$ is the dimension of the finite element basis.
${\color{burgundy}\mathbf{A}}^e$ is the finite element operator for the element and $\mathbf{Q}$ is the localization operator for Fourier modes on an element.
\label{def:high_order_symbol}
\end{definition}

Note that this LFA framework is applicable to any second-order PDE with a weak form that can be represented by the matrix-free representation given by Equation \ref{eq:galerkin_form} or Equation \ref{eq:jacobian_form}.
This representation is used in \href{https://www.github.com/jeremylt/LFAToolkit.jl}{LFAToolkit.jl}, where the users provide the finite element basis ${\color{blue(ncs)}\mathbf{B}}$, the Fourier mode localization operator $\mathbf{Q}$, and the pointwise representation of the weak form ${\color{applegreen}\mathbf{D}}$, and the software provides the LFA of the PDE operator and various preconditioners.

As we develop LFA of various preconditioners and smoothers in this framework throughout this chapter and the following chapters, we will use the scalar diffusion operator as our standard test case to illustrate the properties these techniques with this analysis.

\begin{figure}[!h]
  \centering
  \subfloat[Spectrum of Scalar Diffusion for $p = 4$]{\includegraphics[width=0.48\textwidth]{../img/diffusionSymbol1D}\label{fig:diffusion_spectrum_1d}}
  \hfill
  \subfloat[Spectrum of Scalar Diffusion for $p = 4$]{\includegraphics[width=0.48\textwidth]{../img/diffusionSymbol2D}\label{fig:diffusion_spectrum_2d}}
  \caption{Spectrum of Scalar Diffusion Operator Symbol}
\end{figure}

We see the spectrum of the one dimensional scalar diffusion operator in Figure \ref{fig:diffusion_spectrum_1d} and the spectral radius of the symbol of the two dimensional scalar diffusion operator in Figure \ref{fig:diffusion_spectrum_2d}.
In both cases, we used a fourth-order $H^1$ Lagrange finite element basis on the Gauss-Lobatto points.
These plots were generated with \href{https://www.github.com/jeremylt/LFAToolkit.jl}{LFAToolkit.jl}.
Various preconditioning techniques will reduce the spectral radius of this symbol, each with different effectiveness in different frequency ranges.