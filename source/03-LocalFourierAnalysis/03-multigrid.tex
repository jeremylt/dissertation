With the representation of the symbol of high-order PDE operators given in Definition \ref{def:high_order_symbol}, we can derive the symbol of multigrid error propagation operators.

The total multigrid error propagation operator is given by
\begin{equation}
\mathbf{E}_{\text{TMG}} = {\color{burgundy}\mathbf{S}}_f \left( \mathbf{I} - {\color{burgundy}\mathbf{P}}_{\text{ctof}} {\color{burgundy}\mathbf{A}}_c^{-1} {\color{burgundy}\mathbf{R}}_{\text{ftoc}} A_f \right) {\color{burgundy}\mathbf{S}}_f
\end{equation}
where ${\color{burgundy}\mathbf{S}}_f$ represents the smoother error propagation operator, while ${\color{burgundy}\mathbf{P}}_{\text{ctof}}$ and ${\color{burgundy}\mathbf{R}}_{\text{ftoc}}$ represent the grid prolongation and restriction operators between the coarse and fine grids, respectively.

This error propagation operator can represent both h-multigrid and p-multigrid, depending upon the grid transfer operators and coarse grid representation chosen.

% -- P-Multigrid --------------------------------------------------------------
\subsection{P-Multigrid}

In p-multigrid, grid transfer operators can be represented elementwise and can thus be easily represented in the form of Equation \ref{eq:libceed_representation}.
The prolongation operator from the coarse to the fine grid can be represented by
\begin{equation}
\begin{split}
{\color{burgundy}\mathbf{P}}_{\text{ctof}} = \mathbf{P}_f^T {\color{burgundy}\mathbf{P}}_e \mathbf{P}_c\\
{\color{burgundy}\mathbf{P}}_e = \mathbf{I} {\color{applegreen}\mathbf{D}}_{\text{scale}} {\color{blue(ncs)}\mathbf{B}}_{\text{ctof}}
\end{split}
\end{equation}
where ${\color{blue(ncs)}\mathbf{B}}_{ctof}$ is an interpolation operator from the coarse grid basis to the fine grid basis, $\mathbf{P}_f$ is the fine grid element assembly operator, $\mathbf{P}_c$ is the coarse grid element assembly operator, and ${\color{applegreen}\mathbf{D}}_{\text{scale}}$ is a scaling operator to account for node multiplicity across element interfaces.
Restriction from the fine grid to the coarse grid is given by the transpose, ${\color{burgundy}\mathbf{R}}_{\text{ftoc}} = {\color{burgundy}\mathbf{P}}_{\text{ctof}}^T$.

Following the derivation from Section \ref{sec:lfahighorder}, we can derive the symbols of ${\color{burgundy}\mathbf{P}}_{\text{ctof}}$ and ${\color{burgundy}\mathbf{R}}_{\text{ftoc}}$.

\begin{definition}
The symbol of the p-prolongation is given by
\begin{equation}
\tilde{{\color{burgundy}\mathbf{P}}}_{\text{ctof}} \left( \theta \right) = \mathbf{Q}_f^T \left( {\color{burgundy}\mathbf{P}}_e \odot \left[ e^{\imath \sum_d \left( \mathbf{x}_{i, f} - \mathbf{x}_{j, c} \right) \mathbf{\theta} / \mathbf{h}} \right] \right) \mathbf{Q}_c
\end{equation}
where $i \in \left[ 0, 1, \dots, p_{\text{fine}} \right]$, $h$ is the length of the element, and $j \in \left[ 0, 1, \dots, p_{\text{coarse}} \right]$.
The matrices $\mathbf{Q}_f$ and $\mathbf{Q}_c$ are the localization mappings for the fine and coarse grid, respectively, and the element p-prolongation operator is given by ${\color{burgundy}\mathbf{P}}_e = {\color{applegreen}\mathbf{D}}_{\text{scale}} {\color{blue(ncs)}\mathbf{B}}_{\text{ctof}}$.
\label{def:p_prolongation_symbol}
\end{definition}

\begin{definition}
The symbol of p-restriction is given by the expression
\begin{equation}
\tilde{{\color{burgundy}\mathbf{R}}}_{\text{ftoc}} \left( \theta \right) = \mathbf{Q}_c^T \left( {\color{burgundy}\mathbf{R}}_e \odot \left[ e^{\imath \sum_d \left( \mathbf{x}_{i, c} - \mathbf{x}_{j, f} \right) \mathbf{\theta} / \mathbf{h}} \right] \right) \mathbf{Q}_f
\end{equation}
where $i \in \left[ 0, 1, \dots, p_{\text{coarse}} \right]$, $h$ is the length of the element, and $j \in \left[ 0, 1, \dots, p_{\text{fine}} \right]$.
The matrices $\mathbf{Q}_f$ and $\mathbf{Q}_c$ are the localization mappings for the fine and coarse grid, respectively, and the element p-restriction operator is given by ${\color{burgundy}\mathbf{R}}_e = P_e^T = {\color{blue(ncs)}\mathbf{B}}_{\text{ctof}}^T {\color{applegreen}\mathbf{D}}_{\text{scale}}$.
\label{def:p_restriction_symbol}
\end{definition}

% -- H-Multigrid --------------------------------------------------------------
\subsection{H-Multigrid}

TODO: H MULTIGRID
